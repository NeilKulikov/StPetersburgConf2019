% !TEX encoding = UTF-8 Unicode
\documentclass[9pt,pdf]{beamer}
\usepackage{graphicx}
\usepackage[T2A]{fontenc}
\usepackage[utf8]{inputenc}
\usepackage[russian]{babel}
\usepackage{tikz}
\usepackage{natbib}
\usepackage{tabularx}

\usetikzlibrary{decorations.markings}

\usetikzlibrary{decorations.pathmorphing}

\usetikzlibrary{petri}

\tikzset{snake it/.style={decorate, decoration=snake}}

\title{Расчет порога оже-рекомбинации в узкозонных гетеростуктурах на основе HgCdTe}
\author{Выполнил: Куликов Н.С.\\
        Научные руководители: Морозов С.В., Жолудев М.С.}
\institute{ИФМ РАН\\\vspace{0.5cm}\includegraphics[width=2cm]{images/logo.png}}
\date{2019}

\newcolumntype{Y}{>{\centering\arraybackslash}X}

\tikzset{   main/.style = {rectangle, draw = blue, thick, fill = white, text width = 5 cm, minimum height=2em},
            lower/.style = {rectangle, draw = blue, thick, fill = white, text width = 4 cm, minimum height=2em}}

\begin{document}

    \frame{\titlepage}

    \begin{frame}
        \frametitle{Типы рекомбинации}
        
        \begin{columns}
            \begin{column}{.333\textwidth}
                \begin{center}
                    \textbf{Излучательная}
                \end{center}
            \end{column}
            \hfill
            \begin{column}{.333\textwidth}
                \begin{center}
                    \textbf{Шокли-Рида-Холла}
                \end{center}
            \end{column}
            \hfill
            \begin{column}{.333\textwidth}
                \begin{center}
                    \textbf{Оже}
                \end{center}
            \end{column}
        \end{columns}
        \vfill
        \begin{columns}
            \begin{column}{.333\textwidth}
                \begin{center}
                    \resizebox{0.9\textwidth}{!}{
                      \begin{tikzpicture}
                        \begin{scope}[very thick,decoration={
                          markings,
                          mark=at position 0.5 with {\arrow{>}}}
                          ]
                        \draw[domain=-1.3:1.3,smooth,variable=\x] plot ({\x},{0.5 * \x * \x + 0.75});
                        \draw[domain=-2:2,smooth,variable=\x] plot ({\x},{-0.25 * \x * \x - 0.75});
                        \draw[color=black, fill=blue] (0., 0.75) circle (.1);
                        \draw[postaction={decorate}, blue, thick] (0., 0.75) -- (0., -0.75); 
                        \draw[color=black, fill=red] (0., -0.75) circle (.1);
                        \draw[->, snake it] (0.1, 0.) -- (1.1, 0.);
                        \end{scope}
                      \end{tikzpicture}
                    }
                  \end{center}
            \end{column}
            \begin{column}{.333\textwidth}
                \begin{center}
                    \resizebox{0.9\textwidth}{!}{
                      \begin{tikzpicture}
                        \begin{scope}[very thick,decoration={
                          markings,
                          mark=at position 0.5 with {\arrow{>}}}
                          ]
                        \draw[domain=-1.3:1.3,smooth,variable=\x] plot ({\x},{0.5 * \x * \x + 0.75});
                        \draw[domain=-2:2,smooth,variable=\x] plot ({\x},{-0.25 * \x * \x - 0.75});
                        \draw (-0.5, 0.) -- (0.5, 0.);
                        \draw[color=black, fill=blue] (0., 0.75) circle (.1);
                        \draw[postaction={decorate}, blue, thick] (0., 0.75) -- (0., 0.); 
                        \draw[color=black, fill=red] (0., -0.75) circle (.1);
                        \draw[postaction={decorate}, red, thick] (0., -0.75) -- (0., 0.); 
                        \end{scope}
                      \end{tikzpicture}
                    }
              \end{center}
            \end{column}
            \begin{column}{.333\textwidth}
                \begin{center}
                    \resizebox{0.9\textwidth}{!}{
                      \begin{tikzpicture}
                        \begin{scope}[very thick,decoration={
                          markings,
                          mark=at position 0.5 with {\arrow{>}}}
                          ]
                        \draw[domain=-2:2,smooth,variable=\x] plot ({\x},{0.5 * \x * \x + 0.25});
                        \draw[domain=-2:2,smooth,variable=\x] plot ({\x},{-0.25 * \x * \x - 0.25});
                        \draw[color=black, fill=blue] (0.289, 0.391) circle (.1);
                        \draw[color=black, fill=blue] (0.289, 0.191) circle (.1);
                        \draw[postaction={decorate}, red, thick] (0.289, 0.291) -- (-0.577, -0.333);
                        \draw[postaction={decorate}, blue, thick] (0.289, 0.291) -- (1.15, 0.918); 
                        \draw[color=black, fill=red] (-0.577, -0.333) circle (.1);
                        \draw[color=black, fill=white] (1.15, 0.918) circle (.1);
                        \end{scope}
                      \end{tikzpicture}
                    }
                  \end{center}
            \end{column}
        \end{columns}
        \vfill
        \begin{columns}
            \begin{column}{.333\textwidth}
                Целевой процесс.
            \end{column}
            \hfill
            \begin{column}{.333\textwidth}
                Подавлен в силу малой кон-ии примесей.
            \end{column}
            \hfill
            \begin{column}{.333\textwidth}
                Не может быть подавлен технологическими приёмами.
            \end{column}
        \end{columns}
    \end{frame}

    \begin{frame}
        \frametitle{Порог оже-процессов}
        \begin{columns}
            \begin{column}{0.49\textwidth}
                \begin{center}
                    \begin{tikzpicture}[node distance = 1.5cm, auto]
                        \node [main] (law) {Законы сохранения:
                          \begin{equation*}
                            \begin{aligned}
                            \vec{k}_1 + \vec{k}_2 - \vec{k}_3 = \vec{k}_\text{f};\\
                            \varepsilon_{1}(\vec{k}_{1}) + \varepsilon_{2}(\vec{k}_{2})
                             - \varepsilon_{3}(\vec{k}_{3}) = \varepsilon_\text{f}(\vec{k}_f);
                            \end{aligned}
                          \end{equation*}};
                        \node [main, below of = law] (thr) {Наличие пороговой энергии $\varepsilon_\text{th}$};
                        \node [main, below of = thr] {Равенство групповых скоростей};
                    \end{tikzpicture}
                \end{center}

            \end{column}
            \hfill
            \begin{column}{0.49\textwidth}
                \begin{center}
                    \resizebox{\textwidth}{!}{
                      \begin{tikzpicture}
                        \begin{scope}[very thick,decoration={
                          markings,
                          mark=at position 0.5 with {\arrow{>}}}
                          ]
                        \draw[domain=-2:2,smooth,variable=\x] plot ({\x},{0.5 * \x * \x + 0.25});
                        \draw[domain=-2:2,smooth,variable=\x] plot ({\x},{-0.25 * \x * \x - 0.25});
                        \draw[color=black, fill=blue] (0.289, 0.391) circle (.1);
                        \draw[color=black, fill=blue] (0.289, 0.191) circle (.1);
                        \draw[postaction={decorate}, red, thick] (-0.577, -0.333) -- (0.289, 0.291);
                        \draw[postaction={decorate}, blue, thick] (0.289, 0.291) -- (1.15, 0.918); 
                        \draw[color=black, fill=red] (-0.577, -0.333) circle (.1) node {$\vec{k}_1$};
                        \draw[color=black, fill=white] (1.15, 0.918) circle (.1);
                        \end{scope}
                      \end{tikzpicture}
                    }
                  \end{center}
            \end{column}
        \end{columns}
    \end{frame}

  
\end{document}